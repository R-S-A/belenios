\documentclass[a4paper]{article}
\usepackage{a4wide}
\usepackage[utf8]{inputenc}
\usepackage[T1]{fontenc}
\usepackage{lmodern}
\usepackage{amsmath}
\usepackage{bbm}
\usepackage{hyperref}

\newcommand{\version}{0.1}

\newcommand{\F}{\mathbbm{F}}
\newcommand{\G}{\mathbbm{G}}
\newcommand{\Z}{\mathbbm{Z}}
\newcommand{\N}{\mathbbm{N}}
\newcommand{\I}{\mathbbm{I}}

\newcommand{\public}{\textsf{public}}
\newcommand{\shuffle}{\textsf{shuffle}}
\newcommand{\basesixfour}{\textsf{BASE64}}
\newcommand{\shatwo}{\textsf{SHA256}}

\newcommand{\jstring}{\texttt{string}}
\newcommand{\uuid}{\texttt{uuid}}
\newcommand{\tpk}{\texttt{trustee\_public\_key}}
\newcommand{\election}{\texttt{election}}
\newcommand{\ballot}{\texttt{ballot}}
\newcommand{\etally}{\texttt{encrypted\_tally}}
\newcommand{\pdecryption}{\texttt{partial\_decryption}}
\newcommand{\result}{\texttt{result}}

\title{Belenios specification}
\date{Version~\version}
\author{Stéphane Glondu}

\begin{document}
\maketitle
\tableofcontents

\section{Introduction}

This document is a specification of the voting protocol implemented in
Belenios v\version. More discussion, theoretical explanations and
bibliographical references can be found in a technical report
available online.\footnote{\url{http://eprint.iacr.org/2013/177}}

The Belenios protocol is very similar to Helios (with a signature
added to ballots and different zero-knowledge proofs) and Helios-C
(with the distributed key generation of trustees of Helios, without
threshold support).

The cryptography involved in Belenios needs a cyclic group $\G$ where
discrete logarithms are hard to compute. We will denote by $g$ a
generator and $q$ its order. We use a multiplicative notation for the
group operation. For practical purposes, we use a multiplicative
subgroup of $\F^*_p$ (hence, all exponentiations are implicitly done
modulo $p$). We suppose the group parameters are agreed on
beforehand. Default group parameters are given as examples in
section~\ref{default-group} (they are the same as Helios v3).

\section{Parties}

\begin{itemize}
\item $S$: voting server
\item $A$: server administrator
\item $C$: credential authority
\item $T_1,\dots,T_m$: trustees
\item $V_1,\dots,V_n$: voters
\end{itemize}

\section{Processes}
\label{processes}

\subsection{Election setup}
\label{election-setup}

\begin{enumerate}
\item $A$ generates a fresh \hyperref[basic-types]{$\uuid$} $u$ and
  sends it to $C$
\item $C$ generates \hyperref[credentials]{credentials}
  $c_1,\dots,c_n$ and computes
  $L=\shuffle(\public(c_1),\dots,\public(c_n))$
\item for $j\in[1\dots n]$, $C$ sends $c_j$ to $V_j$
\item $C$ forgets $c_1,\dots,c_n$
\item $C$ forgets the mapping between $j$ and $\public(c_j)$
  if credential recovery is not needed
\item $C$ sends $L$ to $A$
\item for $z\in[1\dots m]$,
  \begin{enumerate}
  \item $T_z$ generates a \hyperref[trustee-keys]{$\tpk$} $k_z$ and
    sends it to $A$
  \item $A$ checks $k_z$
  \end{enumerate}
\item $A$ combines all the trustee public keys into the election
  public key $y$
\item $A$ creates the \hyperref[elections]{$\election$} $E$
\item $A$ loads $E$ and $L$ into $S$ and starts it
\end{enumerate}

\subsection{Vote}

\begin{enumerate}
\item $V$ gets $E$
\item $V$ creates a \hyperref[ballots]{$\ballot$} $b$ and submits it to $S$
\item $S$ validates $b$ and publishes it
\end{enumerate}

\subsection{Credential recovery}

\begin{enumerate}
\item $V$ contacts $C$
\item $C$ looks up $V$'s public credential $\public(c_i)$ and
  generates a new credential $c'_i$
\item $C$ sends $c'_i$ to $V$ and forgets it
\item $C$ sends $\public(c_i)$ and $\public(c'_i)$ to $A$
\item $A$ checks that $\public(c_i)$ has not been used and replaces it
  by $\public(c'_i)$ in $L$
\end{enumerate}

\subsection{Tally}

\begin{enumerate}
\item $A$ stops $S$ and computes the \hyperref[tally]{$\etally$} $\Pi$
\item for $z\in[1\dots m]$,
  \begin{enumerate}
  \item $A$ sends $\Pi$ to $T_z$
  \item $T_z$ generates a \hyperref[tally]{$\pdecryption$} $\delta_z$
    and sends it to $A$
  \item $A$ verifies $\delta_z$
  \end{enumerate}
\item $A$ combines all the partial decryptions, computes and publishes
  the election \hyperref[election-result]{\result}
\end{enumerate}

\section{Messages}
\label{messages}

\subsection{Conventions}

Structured data is encoded in JSON (RFC 4627). There is no specific
requirement on the formatting and order of fields, but care must be
taken when hashes are computed. We use the notation
$\textsf{field}(o)$ to access the field \textsf{field} of $o$.

\subsection{Basic types}
\label{basic-types}

\begin{itemize}
\item $\jstring$: JSON string
\item $\uuid$: UUID (see RFC 4122), encoded as a JSON string
\item $\I$: small integer, encoded as a JSON number
\item $\N$, $\Z_q$, $\G$: big integer, written in base 10 and encoded as a
  JSON string
\end{itemize}

\subsection{Common structures}
\label{common}

\newcommand{\pk}{\texttt{public\_key}}
\newcommand{\sk}{\texttt{private\_key}}
\newcommand{\proof}{\texttt{proof}}
\newcommand{\iproof}{\texttt{iproof}}
\newcommand{\ciphertext}{\texttt{ciphertext}}

\newcommand{\pklabel}{\textsf{public\_key}}
\newcommand{\pok}{\textsf{pok}}
\newcommand{\challenge}{\textsf{challenge}}
\newcommand{\response}{\textsf{response}}
\newcommand{\alphalabel}{\textsf{alpha}}
\newcommand{\betalabel}{\textsf{beta}}
\newcommand{\Hash}{\mathcal{H}}

\begin{gather*}
  \proof=\left\{
    \begin{array}{rcl}
      \challenge&:&\Z_q\\
      \response&:&\Z_q
    \end{array}
  \right\}
  \qquad
  \ciphertext=\left\{
    \begin{array}{rcl}
      \alphalabel&:&\G\\
      \betalabel&:&\G
    \end{array}
  \right\}
\end{gather*}

\subsection{Trustee keys}
\label{trustee-keys}

\begin{gather*}
  \pk=\G\qquad\sk=\Z_q\\
  \tpk=\left\{
    \begin{array}{rcl}
      \pok&:&\proof\\
      \pklabel&:&\pk
    \end{array}
  \right\}
\end{gather*}

A private key is a random number $x$ modulo $q$. The corresponding
$\pklabel$ is $X=g^x$. A $\tpk$ is a bundle of this public key with a
\hyperref[common]{$\proof$} of knowledge computed as follows:
\begin{enumerate}
\item pick a random $w\in\Z_q$
\item compute $A=g^w$
\item $\challenge=\Hash_\pok(X,A)\mod q$
\item $\response=w+x\times\challenge\mod q$
\end{enumerate}
where $\Hash_\pok$ is computed as follows:
\[\Hash_\pok(X,A) = \shatwo(\verb=pok|=X\verb=|=A) \]
where $\pok$ and the vertical bars are verbatim and numbers are
written in base 10. The result is interpreted as a 256-bit big-endian
number. The proof is verified as follows:
\begin{enumerate}
\item compute $A={g^\response}/{y^\challenge}$
\item check that $\challenge=\Hash_\pok(\pklabel,A)\mod q$
\end{enumerate}

\subsection{Credentials}
\label{credentials}

\newcommand{\secret}{\texttt{secret}}

A secret \emph{credential} $c$ is a 15-character string, where characters are
taken from the set:
\[\texttt{123456789ABCDEFGHJKLMNPQRSTUVWXYZabcdefghijkmnopqrstuvwxyz}\]
The first 14 characters are random, and the last one is a checksum to
detect typing errors. To compute the checksum, each character is
interpreted as a base 58 digit: $\texttt{1}$ is $0$, $\texttt{2}$ is
$1$, \dots, $\texttt{z}$ is $57$. The first 14 characters are
interpreted as a big-endian number $c_1$ The checksum is $53-c_1\mod
53$.

From this string, a secret exponent $s=\secret(c)$ is derived by using
PBKDF2 (RFC 2898) with:
\begin{itemize}
\item $c$ as password;
\item HMAC-SHA256 (RFC 2104, FIPS PUB 180-2) as pseudorandom function;
\item the $\uuid$ (interpreted as a 16-byte array) of the election as
  salt;
\item $1000$ iterations
\end{itemize}
and an output size of 1 block, which is interpreted as a big-endian
256-bit number and then reduced modulo $q$ to form $s$.  From this
secret exponent, a public key $\public(c)=g^s$ is computed.

\subsection{Election}
\label{elections}

\newcommand{\question}{\texttt{question}}

\begin{gather*}
  \texttt{wrapped\_pk}=\left\{
    \begin{array}{rcl}
      \textsf{g}&:&\G\\
      \textsf{p}&:&\N\\
      \textsf{q}&:&\N\\
      \textsf{y}&:&\G
    \end{array}
  \right\}
\end{gather*}
The election public key, which is denoted by $y$ thoughout this
document, is computed by multiplying all the public keys of the
trustees, and bundled with the group parameters in a
\texttt{wrapped\_pk} structure.

\newcommand{\minlabel}{\textsf{min}}
\newcommand{\maxlabel}{\textsf{max}}
\newcommand{\answers}{\textsf{answers}}

\begin{gather*}
  \question=\left\{
    \begin{array}{rcl}
      \answers&:&\jstring^\ast\\
      \minlabel&:&\I\\
      \maxlabel&:&\I\\
      \textsf{question}&:&\jstring
    \end{array}
  \right\}
  \qquad
  \election=\left\{
    \begin{array}{rcl}
      \textsf{description}&:&\jstring\\
      \textsf{name}&:&\jstring\\
      \textsf{public\_key}&:&\texttt{wrapped\_pk}\\
      \textsf{questions}&:&\texttt{question}^\ast\\
      \textsf{uuid}&:&\texttt{uuid}\\
      \textsf{short\_name}&:&\jstring
    \end{array}
  \right\}
\end{gather*}

\newcommand{\answer}{\texttt{answer}}
\newcommand{\signature}{\texttt{signature}}
\newcommand{\iproofs}{\textsf{individual\_proofs}}
\newcommand{\oproof}{\textsf{overall\_proof}}
\newcommand{\choices}{\textsf{choices}}
\newcommand{\iprove}{\textsf{iprove}}

During an election, the following data needs to be public in order to
verify the setup phase and to validate ballots:
\begin{itemize}
\item the $\election$ structure described above;
\item all the $\tpk$s that were generated during the
  \hyperref[election-setup]{setup phase};
\item the set $L$ of public credentials.
\end{itemize}

\subsection{Proofs of interval membership}

\begin{gather*}
  \iproof=\proof^\ast
\end{gather*}

Given a pair $(\alpha,\beta)$ of group elements, one can prove that it
has the form $(g^r,y^rg^{M_i})$ with $M_i\in[M_0,\dots,M_k]$ by
creating a sequence of $\proof$s $\pi_0,\dots,\pi_k$ with the
following procedure, parameterised by a group element $S$:
\begin{enumerate}
\item for $j\neq i$:
  \begin{enumerate}
  \item create $\pi_j$ with a random $\challenge$ and a random
    $\response$
  \item compute
    \[A_j=\frac{g^\response}{\alpha^\challenge}\quad\text{and}\quad
    B_j=\frac{y^\response}{(\beta/g^{M_j})^\challenge}\]
  \end{enumerate}
\item $\pi_i$ is created as follows:
  \begin{enumerate}
  \item pick a random $w\in\Z_q$
  \item compute $A_i=g^w$ and $B_i=y^w$
  \item $\challenge(\pi_i)=\Hash_\iprove(S,\alpha,\beta,A_0,B_0,\dots,A_k,B_k)-\sum_{j\neq
      i}\challenge(\pi_j)\mod q$
  \item $\response(\pi_i)=w+r\times\challenge(\pi_i)\mod q$
  \end{enumerate}
\end{enumerate}
In the above, $\Hash_\iprove$ is computed as follows:
\[\Hash_\iprove(S,\alpha,\beta,A_0,B_0,\dots,A_k,B_k)=\shatwo(\verb=prove|=S\verb=|=\alpha\verb=,=\beta\verb=|=A_0\verb=,=B_0\verb=,=\dots\verb=,=A_k\verb=,=B_k)\]
where \verb=prove=, the vertical bars and the commas are verbatim and
numbers are written in base 10. The result is interpreted as a 256-bit
big-endian number. We will denote the whole procedure by
$\iprove(S,r,i,M_0,\dots,M_k)$.

The proof is verified as follows:
\begin{enumerate}
\item for $j\in[0\dots k]$, compute
  \[A_j=\frac{g^{\response(\pi_j)}}{\alpha^{\challenge(\pi_j)}}\quad\text{and}\quad
  B_j=\frac{y^{\response(\pi_j)}}{(\beta/g^{M_j})^{\challenge(\pi_j)}}\]
\item check that
  \[\Hash_\iprove(S,\alpha,\beta,A_0,B_0,\dots,A_k,B_k)=\sum_{j\in[0\dots
    k]}\challenge(\pi_j)\mod q\]
\end{enumerate}

\subsection{Encrypted answers}
\label{answers}

\begin{gather*}
  \answer=\left\{
    \begin{array}{rcl}
      \choices&:&\ciphertext^\ast\\
      \iproofs&:&\iproof^\ast\\
      \oproof&:&\iproof
    \end{array}
  \right\}
\end{gather*}

An answer to a \hyperref[elections]{$\question$} is the vector of
encrypted weights (\choices, same length as \answers) given to each
answer. Each weight comes with a disjunctive proof (in \iproofs, same
length as \choices) that it is indeed 0 or 1. The whole answer also
comes with a proof (\oproof) that the sum of weights is within bounds
$[\minlabel\dots\maxlabel]$.

More concretely, each weight $m\in[0\dots1]$ is encrypted into a
$\ciphertext$ as follows:
\begin{enumerate}
\item pick a random $r\in\Z_q$
\item $\alphalabel=g^r$
\item $\betalabel=y^rg^m$
\end{enumerate}
where $y$ is the election public key.

To compute the proofs, the voter needs a
\hyperref[credentials]{credential} $c$. Let $s=\secret(c)$, and
$S=g^s$ written in base 10. The individual proof that $m\in[0\dots1]$
is computed by running $\iprove(S,r,m,0,1)$.  The overall proof that
$M\in[\minlabel\dots\maxlabel]$ is computed by running
$\iprove(S,R,M-\minlabel,\minlabel,\dots,\maxlabel)$ where $R$ is the
sum of the $r$ used in ciphertexts, and $M$ the sum of the $m$.

\subsection{Signatures}
\label{signatures}

\begin{gather*}
  \signature=\left\{
    \begin{array}{rcl}
      \pklabel&:&\pk\\
      \challenge&:&\Z_q\\
      \response&:&\Z_q
    \end{array}
  \right\}
\end{gather*}

\newcommand{\siglabel}{\textsf{signature}}

Each ballot contains a digital signature to avoid ballot stuffing. The
signature needs a \hyperref[credentials]{credential} $c$ and uses all
the \ciphertext{}s $\gamma_1,\dots,\gamma_l$ that appear in the ballot
($l$ is the sum of the lengths of $\choices$). It is computed as
follows:
\begin{enumerate}
\item compute $s=\secret(c)$
\item pick a random $w\in\Z_q$
\item compute $A=g^w$
\item $\pklabel=g^s$
\item $\challenge=\Hash_\siglabel(\pklabel,A,\gamma_1,\dots,\gamma_l)\mod q$
\item $\response=w-s\times\challenge\mod q$
\end{enumerate}
In the above, $\Hash_\siglabel$ is computed as follows:
\[
\Hash_\siglabel(S,A,\gamma_1,\dots,\gamma_l)=\shatwo(\verb=sig|=S\verb=|=A\verb=|=\alphalabel(\gamma_1)\verb=,=\betalabel(\gamma_1)\verb=,=\dots\verb=,=\alphalabel(\gamma_l)\verb=,=\betalabel(\gamma_l))
\]
where \verb=sig=, the vertical bars and commas are verbatim and
numbers are written in base 10. The result is interpreted as a 256-bit
big-endian number.

Signatures are verified as follows:
\begin{enumerate}
\item compute $A=g^\response\times \pklabel^\challenge$
\item check that $\challenge=\Hash_\siglabel(\pklabel,A,\gamma_1,\dots,\gamma_l)\mod q$
\end{enumerate}

\subsection{Ballots}
\label{ballots}

\newcommand{\json}{\textsf{JSON}}

\begin{gather*}
  \ballot=\left\{
    \begin{array}{rcl}
      \answers&:&\hyperref[answers]{\answer}^\ast\\
      \textsf{election\_hash}&:&\jstring\\
      \textsf{election\_uuid}&:&\uuid\\
      \siglabel&:&\hyperref[signatures]{\signature}
    \end{array}
  \right\}
\end{gather*}
The so-called hash (or \emph{fingerprint}) of the election
is computed with the function $\Hash_\json$:
\[
\Hash_\json(J)=\basesixfour(\shatwo(J))
\]
Where $J$ is the serialization (done by the server) of the $\election$
structure.

The same hashing function is used on a serialization (done by the
voting client) of the $\ballot$ structure to produce a so-called
\emph{smart ballot tracker}.

\subsection{Tally}
\label{tally}

\begin{gather*}
  \etally=\ciphertext^\ast{}^\ast
\end{gather*}
The encrypted tally is the pointwise product of the ciphertexts of all
accepted ballots:
\[
\begin{array}{rcl}
\alphalabel(\etally_{i,j})&=&\prod\alphalabel(\choices(\answers(\ballot)_i)_j)\\
\betalabel(\etally_{i,j})&=&\prod\betalabel(\choices(\answers(\ballot)_i)_j)
\end{array}
\]

\newcommand{\dfactors}{\textsf{decryption\_factors}}
\newcommand{\dproofs}{\textsf{decryption\_proofs}}
\newcommand{\decrypt}{\textsf{decrypt}}

\begin{gather*}
  \pdecryption=\left\{
    \begin{array}{rcl}
      \dfactors&:&\G^\ast{}^\ast\\
      \dproofs&:&\proof^\ast{}^\ast
    \end{array}
  \right\}
\end{gather*}
From the encrypted tally, each trustee computes a partial decryption
using the \hyperref[trustee-keys]{private key} $x$ (and the
corresponding public key $X=g^x$) he generated during election
setup. It consists of so-called decryption factors:
\[
\dfactors_{i,j}=\alphalabel(\etally_{i,j})^x
\]
and proofs that they were correctly computed. Each $\dproofs_{i,j}$ is
computed as follows:
\begin{enumerate}
\item pick a random $w\in\Z_q$
\item compute $A=g^w$ and $B=\alphalabel(\etally_{i,j})^w$
\item $\challenge=\Hash_\decrypt(X,A,B)\mod q$
\item $\response=w+x\times\challenge\mod q$
\end{enumerate}
In the above, $\Hash_\decrypt$ is computed as follows:
\[
\Hash_\decrypt(X,A,B)=\shatwo(\verb=decrypt|=X\verb=|=A\verb=,=B)
\]
where \verb=decrypt=, the vertical bars and the comma are verbatim and
numbers are written in base 10. The result is interpreted as a 256-bit
big-endian number.

These proofs are verified using the $\tpk$ structure $k$ that the
trustee sent to the administrator during the election setup:
\begin{enumerate}
\item compute
\[
A=\frac{g^\response}{\pklabel(k)^\challenge}
\quad\text{and}\quad
B=\frac{\alphalabel(\etally_{i,j})^\response}{\dfactors_{i,j}^\challenge}
\]
\item check that $\Hash_\decrypt(\pklabel(k),A,B)=\challenge\mod q$
\end{enumerate}

\subsection{Election result}
\label{election-result}

\newcommand{\ntallied}{\textsf{num\_tallied}}
\newcommand{\etallylabel}{\textsf{encrypted\_tally}}
\newcommand{\pdlabel}{\textsf{partial\_decryptions}}
\newcommand{\resultlabel}{\textsf{result}}

\begin{gather*}
  \result=\left\{
    \begin{array}{rcl}
      \ntallied&:&\I\\
      \etallylabel&:&\etally\\
      \pdlabel&:&\pdecryption^\ast\\
      \resultlabel&:&\I^\ast{}^\ast
    \end{array}
  \right\}
\end{gather*}
The decryption factors are combined for each ciphertext to build
synthetic ones:
\[
F_{i,j}=\prod_{z\in[1\dots m]}\pdlabel_{z,i,j}
\]
where $m$ is the number of trustees. The $\resultlabel$ field of the
$\result$ structure is then computed as follows:
\[
\resultlabel_{i,j}=\log_g\left(\frac{\betalabel(\etallylabel_{i,j})}{F_{i,j}}\right)
\]
Here, the discrete logarithm can be easily computed because it is
bounded by $\ntallied$.

After the election, the following data needs to be public in order to
verify the tally:
\begin{itemize}
\item the $\election$ structure;
\item all the $\tpk$s that were generated during the
  \hyperref[election-setup]{setup phase};
\item the set of public credentials;
\item the set of ballots;
\item the $\result$ structure described above.
\end{itemize}

\section{Default group parameters}
\label{default-group}

\[
\begin{array}{lcr}
p&=&16328632084933010\\
&&002384055033805457329601614771185955389739167309086214800406\\
&&465799038583634953752941675645562182498120750264980492381375\\
&&579367675648771293800310370964745767014243638518442553823973\\
&&482995267304044326777047662957480269391322789378384619428596\\
&&446446984694306187644767462460965622580087564339212631775817\\
&&895958409016676398975671266179637898557687317076177218843233\\
&&150695157881061257053019133078545928983562221396313169622475\\
&&509818442661047018436264806901023966236718367204710755935899\\
&&013750306107738002364137917426595737403871114187750804346564\\
&&731250609196846638183903982387884578266136503697493474682071\\
g&=&14887492224963187\\
&&634282421537186040801304008017743492304481737382571933937568\\
&&724473847106029915040150784031882206090286938661464458896494\\
&&215273989547889201144857352611058572236578734319505128042602\\
&&372864570426550855201448111746579871811249114781674309062693\\
&&442442368697449970648232621880001709535143047913661432883287\\
&&150003429802392229361583608686643243349727791976247247948618\\
&&930423866180410558458272606627111270040091203073580238905303\\
&&994472202930783207472394578498507764703191288249547659899997\\
&&131166130259700604433891232298182348403175947450284433411265\\
&&966789131024573629546048637848902243503970966798589660808533\\
q&=&61329566248342901\\
&&292543872769978950870633559608669337131139375508370458778917
\end{array}
\]

\end{document}
